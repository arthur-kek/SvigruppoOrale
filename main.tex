\documentclass[a4paper,12pt,titlepage,oneside]{book}
\linespread{1.2}
\pagestyle{plain}
\usepackage[italian]{babel} 
\usepackage[utf8]{inputenc}
\usepackage[hidelinks]{hyperref}
\usepackage{graphicx}
\usepackage{xcolor}
\usepackage[affil-it]{authblk}
\usepackage{listings}
\usepackage{float}
\usepackage{caption}
\usepackage{geometry}
\newgeometry{vmargin={25mm}, hmargin={15mm,17mm}}   % set the margins
\usepackage{titlesec, blindtext, color}
\definecolor{gray75}{gray}{0.75}
\newcommand{\hsp}{\hspace{20pt}}
\titleformat{\chapter}[hang]{\Huge\bfseries}{\thechapter\hsp\textcolor{gray75}{|}\hsp}{0pt}{\Huge\bfseries}


\begin{document}
\section{Introduzione}
Mentre nei corsi di programmazione solitamente c'è 1 programmatore che scrive algoritmi per una sola macchina, in un contesto più grande ci sono molti programmatori, suddivisi in team, che rispondono a vari committenti e realizzano sistemi e componenti che devono essere anche mantenuti.

È necessario quindi organizzare lo sviluppo in un gruppo di lavoro complesso.

\section{The mythical man-month}
Nel suo libro "The mythical man-month", Brooks, racconta la sua esperienza nello sviluppo di OS/360. Un progetto pubblicato con un anno di ritardo e con sforamenti di budget altissimi.
Nel libro si parla di come negli anni 70 era molto comune finire nella cosidetta tar-pit, sia con un team piccolo che uno enorme. Ci si ritrovano tutti senza capire la causa e molteplici fattori simultanei portano alla discesa. Tutti se ne possono accorgere ma non se ne capisce la causa.

\subsection{La tar-pit}
Si parte spiegando che per passare da un semplice programma a un prodotto veramente usabile ci vuole almeno lo sforzo originale moltiplicato per 9.
Ogni programmatore si diverte creando la sua piccola opera, tuttavia i problemi arrivano quando deve interagire con altri, ordini dall'alto e altri programmatori. Mentre creare è divertente, trovare e risolvere i bug è solo lavoro. Il prodotto finale spesso non sembra bello come dovrebbe.

\subsection{Legge di Brooks}
Perché un progetto non rispetta i tempi prestabiliti?
- Le tecniche di stima sono sempre ottimiste rispetto alla realtà
    In un progetto ci sono molte attività, spesso collegate una all'altra, che possono andare nel modo sbagliato, aumentando di molto la probabilità di fallire.
- Si confonde sforzo con progresso, personale e mesi non sono intercambiali (il mitico mese-uomo)
- Le stime sono incerte
- Il progresso è monitorato in modo superficiale
- Si tende ad aggredire i ritardi aggiungendo personale, che causa ancora più ritardo

Il mese-uomo non può essere usato come unità di misura perche personale e mesi non sono intercambiali quando si parla di programazione. Il mese-uomo è valido solo per le attività che non necessitano di comunicazione tra il personale (e.g. raccogliere pomodori). È impossibile applicarlo ad attività che non possono essere partizionate su più persone e per attività che richiedono comunicazione complessa come lo sviluppo software può generare risultati anche peggiori.
Ogni risorsa umana aggiunta deve essere formata e questa attività non può essere partizionata, lo sforzo di comunicazione cresce di n(n-1)/2 ogni volta che ne viene aggiunta una.
Il risultato è che aggiungendo risorse si arriva a un punto dove il tempo totale viene allungato e non ridotto.
Legge di Brooks: aggiungendo personale a un progetto di sviluppo software in ritardo, lo farà tardare ancora di più.

\subsection{La sala operatoria}
Tra i programmatori se ne trova sempre uno che è 10 volte più capace e performante degli altri. Si potrebbe pensare di prendere solo questi ultimi e metterli a lavorare in un team piccolo, tuttavia non sarebbe possibile per progetti più grandi.

La sala operatoria di Mills:
- Il chirurgo: molto esperto in più settori, definisce le specifiche, progetta il programma, lo scrive e lo documenta. Conduce i test e controlla le versioni del programma.
- Il copilota: ha meno esperienza del chirurgo ma potrebbe sostituirlo, agisce come consigliere del chirugo. Non scrive il codice ma lo conosce molto bene e fa da intermediatore con il team.
- L'amministratore: si occupa di personale, paga, spazi per conto del chirurgo.
- L'editore: si occupa di revisionare la documentazione scritta dal chirurgo.
- Due segretari: uno per l'amministratore e uno per l'editore, il primo si occupa anche della corrispondenza.
- L'addetto al programmma: si occupa di mantenere i record del team, gestisce i file, mantiene gli input e gli output.
- Il fabbro: si occupa di utities, librerie e degli strumenti che verranno usati per programmare, togliendo questo pensiero a chi programma.
- Il tester: pianifica i test e mette alla prova i programmi.
- L'avvocato del linguaggio: mentre il chirugo pensa alla rappresentazione del programma, ci sarà un altro che conosce ogni segreto e trucco del linguaggio di programmazione per risolvere al meglio un problema. Può essere condiviso da più team.

L'intero team della sala operatoria agisce e pensa come una sola persona.

\subsection{La cattedrale}
Non è possibile scalare la sala operatoria e mantenere l'integrità del sistema. Perché ciò accada, bisogna separare l'architettura dall'implementazione. L'architetto deve occuparsi solo dell'architettura e deve esserci una distinzione netta.

L'integrità concettuale in un progetto è fondamentale, bisogna avere un solo set di idee anziché idee distinte e scoordinate.

La semplicità arriva dall'integrità concettuale, questa arriva da una o poche menti. Tuttavia in un grande progetto deve esserci altra forza lavoro, e questa sarà nettamente separata da chi propone l'architettura. Una volta stabilito cosa si deve fare, si cerca di capire come farlo.

Problemi di questo approccio sono che:
- Le specifiche diventano troppo ricche di funzionalità e costose
- Gli architetti diventano troppo creativi e traciano tutta l'inventività degli implementatori
- Gli implementatori rimangono senza far niente per colpa del collo di bottiglia delle specifiche

\subsection{L'effetto del secondo sistema}
L'architetto deve considerare che l'implementazione potrebbe non andare come pensava e quindi potrebbe superare i costi previsti. L'architetto non ha l'ultima parola sull'implementazione ma deve esserci dialogo.

Di solito il primo lavoro di un architetto è preciso e pulito, non sapendo cosa sta facendo procede cauto e si limita, spesso mette da parte cose troppo ambizione per una seconda volta. Il primo lavoro finisce e l'architetto acquisisce confidenza.

Il secondo lavoro quindi diventa pieno di funzionalità ambiziose mentre si tenta di generalizzare quanto appreso dal lavoro precedente.

\subsection{Passaparola}
Per far si che un gruppo di 10 architetti mantenga l'integrità concettuale di un sistema costruito da 1000 persone è necessario che tutti si riescano a capire.

Il manuale descrive cosa vede e cosa può fare l'utente, non specifica cosa accade dietro e ciò che l'utente non vede. Il manuale deve essere chiaro, preciso e completo, ogni definizione deve contenere tutte le informazioni anche se ripetitive perché l'utente legge solo le parti che gli servono.

\subsection{La torre di Babele}
La torre di Babebe è fallita per mancanza di comunicazione e di conseguenza organizzazione.

Nello sviluppo di un grande progetto è necessaria comunicazione di tre tipi:
- Informale: interpretazione di cosa è scritto
- Riunioni: regolari riunioni per risolvere dubbi
- Workbook: scritto dall'inizio

Il workbook è la struttura che devono avere i documenti che il progetto produce. Deve essere aggiornato e mantenuto.

\section{I modelli a Bazaar}


\end{document}